\documentclass{article}
\usepackage{graphicx}

\title{
\LARGE COMPSCI-1DM3: Assignment \#1 \\
}
\author{\Large Author: Qusay Qadir \\
\large Instructor: Mahdee Jodayree \\ MacID: qadirq \\Tut: T02 \\ }


\date{Due Date: July 7th, 2023}

\begin{document}
\maketitle
\newpage
\tableofcontents
\newpage
\section{Question \#1. [10 Marks]}
Create a truth table for each of the following compound propositions. \\ \\ 
a) ( $q \rightarrow \neg p$ )  $\leftrightarrow$ ( $p \leftrightarrow q$) \\ \\
\begin{table}[h]
\centering
\begin{tabular}{cccccc}
\hline
p & q & $\neg p$ & $(q \rightarrow \neg p)$ & $p \leftrightarrow q$ & $(q \rightarrow \neg p) \leftrightarrow (p \leftrightarrow q)$ \\ 
\hline
1 & 1 & 0 & 0 & 1 & 0 \\
1 & 0 & 0 & 0 & 0 & 1 \\
0 & 1 & 1 & 1 & 0 & 1 \\
0 & 0 & 1 & 1 & 1 & 0 \\
\hline
\end{tabular}
\end{table}
\\ \\ \\ \\ 

b)  ( $q \leftrightarrow p$ )  $\oplus$ ( $p \leftrightarrow \neg q$) \\ \\ 
\begin{table}[h]
\centering
\begin{tabular}{cccccc}
\hline
p & q & $q \leftrightarrow p$ & $p \leftrightarrow \neg q$ & $(q \leftrightarrow p) \oplus (p \leftrightarrow \neg q)$ \\
\hline
1 & 1 & 1  & 0 & 1 \\
1 & 0 & 0 & 1 & 1 \\
0 & 1 & 0 & 0 & 0 \\
0 & 0 & 1 & 1 & 0 \\
\hline
\end{tabular}
\end{table}

\newpage
\section{Question \#2. [10 Marks]}
Express these system specifications using logical connectives (including negations) and the following propositions.
\\\\
\textit{\textbf p}:“The user enters a valid password,” \\
\textit{\textbf q}:“Access is granted,” \\
\textit{\textbf r}:“The user has paid the subscription fee.” \\ \\ 
\textbf {a)} “The user has paid the subscription fee but does not enter a valid password.” \\\\
\textbf {b)} “Access is granted whenever the user has paid the subscription fee and enters a valid password.”\\\\
\textbf {c)} “Access is denied if the user has not paid the subscription fee.”\\\\
\textbf {d)}“If the user has not entered a valid password but has paid the subscription fee, then access is granted.”\\\\
\newpage
\section{Question \#3. [10 Marks]}
For each function, determine whether that function is and $\Omega(x)$ and weather it is (x)\\ \\
a) f(x) = 10 \\ \\ 
b) f(x) = 3x + 7 \\ \\ 
c) f(x) = $x^2$ + x + 1 \\  \\ 
d) f(x) = 5$\log(x)$ \\ \\ 
\newpage

\section{Question \#4. [10 Marks]}
Show that $\frac{x^3 + 2}{2x + 1}$ is \emph{O}($x^2$).
\newpage
\section{Question \#5. [10 Marks]}

Show that each of these pairs of functions is of the same order. \\ \\
a) 3x + 7, x \\\\
b) 2$x^2$ + x - 7, $x^2$ \\ \\ 
c) $\log(x^2 + 1 )$, $\log_{2}{x}$ \\ \\ 
d) $\log_{10}{x}$, $\log_{2}{x}$
\newpage 
\section{Question \#6. [20 Marks]}

Show that $x^5$$y^3$ + $x^4$$y^4$ + $x^3$$y^5$ is $\Omega(x^3y^3)$
\newpage
\section{Question \#7. [20 Marks]}
Consider the following algorithm, which takes as input a sequence of \emph{n} integers \emph{a1,a2,...., an} and produces as output a matrix \textbf{M} = {mij} where \emph{mij} is the minimum term in the sequence of integers \emph{ai,ai+1,...., aj} for \emph {$j \geq i$} and \emph {mij} = 0 otherwise. \\ \\ \indent initialize \textbf {M} so that \emph{mij = ai, if $j \geq i$} and \emph{mij = 0 }otherwise \\ \\ \indent \textbf{for} \emph{i} := 1 \textbf {to} n \\ 
\indent \indent \textbf {for} \emph{j} := \emph {i} + 1 \textbf{to} \emph{n} \\ 
\indent \indent \indent \textbf {for} \emph{k} := \emph {i} + 1 \textbf{to} \emph{j} \\ 
\indent \indent \indent \indent \emph{mij} := min\emph{mij, ak} \\ \\ 
\indent \textbf{return M} = {mij} {\emph{mij} is the minimum term of \emph{ai, ai+1, ..., aj}} \\ \\ 
a) Show that this algorithm uses \emph{O}($n^3$) comparisons to compute the matrix \textbf {M} \\ \\ 
b) Show that this algorithm uses $\Omega(n^3)$ comparisons to compute the matrix \textbf {M}. Using the fact and part (a), conclude that the algorithms uses ($n^3$) comparisons. \\ \\ \emph{Hint:} Only consider the cases where \emph{$i \leq n/4$} and \emph {$j \leq 3n/4$} in the two outer loops in the algorithm.] 
\newpage
\section{Question \#8. [30 Marks]}
What is the effect in the time required to solve a problem when you double the size of the input from \emph{n} to 2\emph{n}, assuming that the number of milliseconds the algorithm uses to solve the problem with input size \emph{n} is each of these functions? [Express your answer in the simplest form possible, either as a ratio or difference. Your answer may be a function of \emph{n} or a constant.] \\ \\  
a) $\log\log{n}$ \\ \\ 
b) $\log(n)$ \\ \\ 
c) 100\emph{n} \\ \\ 
d) \emph{n} $\log{n}$ \\ \\ 
e) $n^2$ \\ \\ 
f)  $n^3$ \\ \\ 
g) $2^n$ \\ \\ 
\newpage
\section{Question \#9. [30 Marks]}
Give a big-\emph{O} estimate for each of these functions. \\
For the function of \emph{g} in your estimate \emph{f(x)} is \emph{O(g(x))}, use a simple function \emph{g} of smallest order. \\ \\ 
a) ($n^3 + n^2log$ n)($\log{n}$ + 1) + (17 $\log{n}$ + 19) ($n^3$ + 2) \\ \\ 
b) ($2^n + n^2$)($n^3 + 3^n$) \\ \\ 
c) ( $n^n + n2^n + 5^n$) (n! + $5^n$ ) \\ \\ 

\newpage

\section{Question \#10. [10 Marks]} 
Use the insertion sort to sort d, f, k, m, a, b, showing the lists obtained at each step. 
\newpage
\section{Question \#11. [20 Marks]}
Write the selection sort algorithm in pseudocode. \\
The selection sort beings by finding the least element in the list. \\
This element is moved to the front. \\ 
Then the least element among the remaining elements is found and put into the second position. \\
This procedure is repeated until the entire list has been sorted. 
\newpage
\section{Questions \#12. [20 Marks]} 
Devise an algorithm that finds all terms of a finite sequence of integers that are greater than the sum of all previous terms of the sequence. 
\end{document}