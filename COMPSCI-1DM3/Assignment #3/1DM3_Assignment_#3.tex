\documentclass{article}
\usepackage{graphicx}

\title{
\LARGE COMPSCI-1DM3: Assignment \#3 CH 6-9 \\
}
\author{\Large Author: Qusay Qadir \\
\large Instructor: Mahdee Jodayree \\ Student \#: 400454763 \\MacID: qadirq  \\Tut: T02 \\ }


\date{Due Date: July 30th(23:59), 2023}

\begin{document}
\maketitle
\newpage
\tableofcontents
\newpage

\section{Question \#1. [30 Marks]}
How many string of eight uppercase English Letters are there \\\\
a) That starts with X if no letter can be repeated \\\
You would need to use the product rule, where the first letter must be an X. This would be expressed as: \\\\
1 $\cdot$ 25 $\cdot$ 24 $\cdot$ 23 $\cdot$ 22 $\cdot$ 21 $\cdot$ 20 $\cdot$ 19 = 2,422,728,000 \\\\
b) That start and end with X, if the letter can be repeated.  \\\\
Since you need to create a string of 8 but this time only the middle 6 can not be X's this would mean that there are \\\\
$26^6$ strings. \\\\
c) Start with letters BO(in that order) if the letters can be repeated? \\\\
There are only 6 letters that can be switched around thus the answer to this is also $26^6$ due to the product rule.
\newpage
\section{Question \#2. [20 Marks]}
How many bit strings of length 12 contain \\ \\
a) At least three 1s? \\\\
At least 3 means that it can also contain 3-12 ones, to solve this we will consider the scenarios where it contradicts this condition and subtract it from \\ $2^{(12)}$  = 4096.
 \\The condition would be violated if there is 2 1's, 1 1's or 0 1's. 
 Thus: \\$2^{(12)} -  (_{0}^{12}) -  (_{1}^{12}) -  (_{2}^{12})$ = 4017 \\\\
 b) An equal number of 0s and 1s? \\\\
 In the following assignment whenever C(n,k) is mentioned assume: 
 \equation{}
 \frac{n!}{k! \cdot (n-k)!}
 \equation{}
 C(12,6) = 924

\newpage

\section{Question \#3. [20 Marks]}
A coin is flipped eight times and each flip comes up either heads or tails. How many possible outcomes? \\\\
a) Contain at least 3 heads? \\ \\ 
Containing at least three heads, like Q2 a), this means there can be 3-8 heads \\ \\
$2^{(8)} -  (_{0}^{8}) -  (_{1}^{8}) -  (_{2}^{8})$ = 219 possible outcomes \\\\
b) Contain the same number of heads and tails? \\\\
That must mean from 8 elements we have to choose from there must be 4 elements that are heads and 4 that are tails: Thus \\ \\
C(8,4) = 70 possible outcomes
\newpage

\section{Question \#4. [10 Marks]}
Find the probability of winning a lottery by selecting the correct six integers, where the order in which these integers are selected does not matter, from the positive integers not exceeding \\ \\
We will use the following equations: \\
P(E) = $\frac{E}{S}$\\\\
a) 42 \\\\
Probability is: $\frac{1}{C(42,6)}$ = 1 / 5245786  \\\\
b) 48 \\\\ 
Probability is: $\frac{1}{C(48,6)}$ = 1 / 12271512
\newpage


\section{Question \#5. [20 Marks]}
What is the probability of these events when we randomly select a permutation of \{ 1,2, ..., n\} where n $\geq$ 4?\\\\
a) 1 immediately precedes 2 \\\\
The probability is 1/2 this is because 1 must precede 2 or follow it, since only one of two events can happen thus the probability of 1 immediately preceding 2 is 1/2.  \\\\
b) n precedes 1 and n-1 precedes 2 \\\\
If we assume a simple case of n = 4 then we can assume that half of the permutations are n preceding 1 and n-1 of them preceding 2. Thus probability is 1/4 since a total of 4 different events can happen that all have an equal chance of occurring. \\\\
c) n precedes 1 and n precedes 2 \\\\
n can either precedes just 1 or just 2 or both thus of the three possible events the probability that 1 and 2 are both preceded by n is 1/3.

\newpage

\section{Question \#6. [20 Marks]}
Suppose that 4\% of the patients tested in a clinic are infected with avian influenza. Furthermore, suppose that when a blood test for avian influenza is given, 97\% of the patients infected with avian influenza test positive and that 2\% of the patients not infected with avian influenza test positive. What is the probability that \\\\
a) a patient testing positive for avian influenza with this test is infected with it? \\\\
Let p(A) = 0.04 and let p($\overline{\mbox{A}}$) = 0.96.
Let  p($\overline{\mbox{P}}$ $|$ A) = 0.03 and let p($\overline{\mbox{P}}$ $|$ $\overline{\mbox{A}}$) = 0.98 \\\\
Use the formula:\\\\ p(A $| $ P) = $\frac{p(P|A)p(A)}{p(P | A)p(A)+ p(P | \overline{\mbox{A}})p(\overline{\mbox{A}})}$ \\\\\\= $\frac{(0.97)(0.04)}{(0.97)(0.04)+(0.02)(0.96)}$ = 0.67 \\\\
b) a patient testing positive for avian influenza with this test is not infected with it? \\\\
p($\overline{\mbox{A}}$ $|$ P) = 1 - p(A $|$ P) = 0.33
\newpage

\section{Question \#7. [30 Marks]}
Let X and Y be the random variables that count the number of heads and the number of tails that come up when two fair coins are flipped. Show that X and Y are not independent. \\\\
Let X be the number of heads that show up and let Y be the number of tails that show up. We know that 2 events are independent if p(X and Y ) = p(X)p(Y) and to show that they are not independent but rather dependent we need to show that p(X and Y) != p(X)p(Y) \\\\
First: p(X=2) = 1/4 and p(Y=2) = 1/4 thus p(X)p(Y) = 1/16 \\\
Second: p(X=2 and Y=2) = 0  \\\\
Since p(X and Y) != p(X)p(Y) thus they are not independent. This completes the proof.
\newpage

\section{Question \#8. [20 Marks]}
a) Find a recurrence relation for the number of ways to climb n stairs if the person climbing the stairs can take one, two, or three stairs at a time. \\\\
Let $a_{(n)}$ be the number of ways to climb up the stairs. To express a person being able to take one stair that person can start at the first step and then climb n-1 stairs in $a_{(n-1)}$ ways, or start by taking two steps and then climb n-2 stairs in $a_{(n-1)}$ ways or start by taking three stairs and then climb n-3 stairs in $a_{(n-1)}$ ways. Thus the recurrence relation is $a_{(n)}$ =  $a_{(n-1)}$ + $a_{(n-1)}$ + $a_{(n-1)}$   \\\\
b) What are the initial conditions? \\\\
$a_{(0)}$ = 1 one way to climb one stair \\ 
$a_{(1)}$ = 1 climb one step at a time twice to climb 2 steps \\ 
$a_{(2)}$ = 2  climb two steps at one time. \\
\newpage

\section{Question \#9 [30 Marks]} 
Find a closed form for the generating function for each of these sequences. (Assume a general form for the terms of the sequence, using the most obvious choice of such a sequence.) \\ \\ 
a) 0, 0, 3, -3, 3, -3, 3, -3, ... \\\\\\\\\\\\\\\\\\\\\\\\\\\\\\\\\\\\\\\\\\\\
b) 1, 2, 1, 1, 1, 1, 1, 1, 1,... \\\\\\\\\\\\\\\\\\\\\\\\\\\\\\\\\\\\
c) $(_{0}^{7}), 2(_{1}^{7}), 2^2(_{2}^{7}),...,2^7(_{0}^{7}), 0, 0, 0, 0, ...$
\newpage

\section{Question \#10. [20 Marks]}
Find the coefficient of $X^{12}$ in the power series of each of these functions.  \\\\
The formulas used are from Table 1: Final Exam- Axioms \\\\
C(n,k) is mentioned assume: 
 \equation{}
 \frac{n!}{k! \cdot (n-k)!}
 \equation{}\\\\
a) $\frac{1}{(1+x)^8}$ \\\\
We will use the formula given by the axiom table: \\ 
$(-1)^k$C(n+k-1, k) = $(-1)^{(12)}$C(19, 7) = 50,388 \\
 \\\\
b) $\frac{1}{(1-4x)^3}$ \\\\
We will use the formula C(n+k-1,k)$a^k$ = C(14,2)$4^{(12)}$ = 1526726656
\newpage
\section{Questions \#11. [20 Marks]} 
Use generating functions to solve the recurrence relation $a_{k} = 7a_{k-1}$ with the initial conditions $a_{0} = 5$
\newpage
\section{Question \#12. [30 Marks]}
Determine whether the relation $R$ on the set of all real numbers is reflexive, symmetric, antisymmetric, and/or transitive, where (x,y) is an element of $R$ if and only if \\\\
a) x = 2y \\\\\\\\\\\\\\\\\\\\\\\\\\\\\\\\\\\\\\\\
b) xy $\geq$ 0
\end{document}




















