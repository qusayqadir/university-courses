\documentclass{article}
\usepackage{graphicx}

\title{
\LARGE COMPSCI-1DM3: Assignment \#2 CH 4-5 \\
}
\author{\Large Author: Qusay Qadir \\
\large Instructor: Mahdee Jodayree \\ MacID: qadirq \\Tut: T02 \\ }


\date{Due Date: July 21st, 2023}

\begin{document}
\maketitle
\newpage
\tableofcontents
\newpage

\section{Question \#1. [30 Marks]}
Suppose that \textbf{a} and \textbf{b} are integers, \textbf{\emph{a}  $\equiv$ 11(mod 19),}  and \textbf{\emph{b }$\equiv$ 3(mod 19)}. Find the integer \textbf{\emph{c}} with 0 $\leq$ c $\leq$ 18 such that \\\\
First we need to determine the value of a, which can be done by finding the remainder of 11 / 19. This would be as such that 
11 = 19(0) + 11. Thus the lowest non-negative value of a that satisfies a $\equiv$ 11(mod 19) is 11 \\\\
Second, we need to determine the value of b, which can be done so as finding the remainder of 3 / 19 which is 3 = 19(0) + 3. Thus the lowest non-negative value of b that satisfies b $\equiv$ 3(mod 19) is 3.\\\\
a) c $\equiv$ 13a(mod 19) \\  \\
c $\equiv$ (13)(11)(mod19)\\\\
c $\equiv$ 143(mod 19) \\\\
Therefore, the value of c is 10 such that the remainder of 143 / 19 is 10,  143 = 19(7) + 10 \\\\
c $\equiv$ 10(mod 19) \\\\
b) c $\equiv$ 8b(mod 19) \\\\
c $\equiv$ (8)(3)(mod19)\\\\
c $\equiv$ 24(mod 19) \\\\
Therefore, the value of c is 5 such that the remainder of 24/ 19 is 5,  24 = 19(0) + 5  \\\\
c $\equiv$ 5(mod 19) \\\\
c) c $\equiv$ a - b(mod 19) \\\\
c $\equiv$ (11 - 3)(mod19)\\\\
c $\equiv$ 8(mod 19) \\\\
Therefore, the value of c is 8 such that the remainder of 9 / 19 is 8,  8 = 19(0) + 8 \\\\
c $\equiv$ 8(mod 19) \\\\
d) c $\equiv$ 7a + 3b(mod 19) \\\\
c $\equiv$ 7(11) + 3(3)(mod19)\\\\
c $\equiv$ 86(mod 19) \\\\
Therefore, the value of c is 10 such that the remainder of 86 / 19 is 10,  86 = 19(4) + 10 \\\\
c $\equiv$ 10(mod 19) \\\\
e) c $\equiv$ 2$a^2$ + 3$b^2$(mod 19) \\\\
c $\equiv$ 2$(11)^2$ + 3$(3)^2$(mod19)\\\\
c $\equiv$ 269(mod 19) \\\\
Therefore, the value of c is 8 such that the remainder of 269 / 19 is 3,  269 = 19(14) + 3 \\\\
c $\equiv$ 3(mod 19) \\\\
f) c $\equiv$ $a^3$ + 4$b^3$(mod 19) \\\\
c $\equiv$ $(11)^3$ + 4$(3)^3$(mod19)\\\\
c $\equiv$ 1439(mod 19) \\\\
Therefore, the value of c is 14 such that the remainder of 1439 / 19 is 14,  1439 = 19(75) + 14 \\\\
c $\equiv$ 14(mod 19) \\\\
\newpage
\section{Question \#2. [20 Marks]}
What are the quotient and remainder when \\\\
To solve the following questions take into consideration the division algorithm. \\ 
a = dq + r where d represents the divisor, q represents the quotient, and r represents the remainder with r being 0 $\leq$ r $<$ d \\\\
a) 19 is divided by 7 \\\\
19 = 7(2) + 5 \\\\
2  = 19 div 7 \\ \\
5 = 19 mod 7 \\ \\ 
b) -111 is divided by 11 \\\\
-111 = 11(-11) + 10 \\\\
-11 = -111 div 11 \\ \\
10  = -111 mod 11  \\\\
c) 789 is divided by 23 \\\\
789 = 23(34) + 7 \\\\
34 = 789 div 23 \\ \\
7 = 789 mod 23  \\\\
d) 1001 is divided by 13 \\\\
1001 = 13(77) + 0 \\\\
77 = 1001 div 13 \\ \\
0  = 1001 mod 13  \\\\
\newpage

\section{Question \#3. [30 Marks]}
Find all the solutions of the congruence \textbf{ $x^2$ $\equiv$ 16(mod 105)} \\ 
The prime factorization of 105 is 3 * 5 * 7. Now we can solve the congruence modulo for each prime factor separately. \\\\
For modulo 3: \\
The solutions are x $\equiv$ 1(mod 3) and x $\equiv$ 2(mod 3) where x = 1 and x = 2 \\\\\\
For modulo 5: \\ 
The solutions are x $\equiv$ 1(mod 5) and x $\equiv$ 4(mod 5) where x = 1 and x = 4 \\\\
For modulo 7: \\ 
The solutions are x $\equiv$ 3(mod 7) and x $\equiv$ 4(mod 7) where x = 3 and x = 4 \\\\\
Since we only need to find 3 solutions of the 8, we only need to use 3 of the possibilities of the Chinese remainder theorem. \\\\
The first system for the Chinese remainder theorem looks like this:  
\begin{center}
1(mod 3)\\
1(mod 5) \\
3(mod 7) \\
\end{center}
The second system for the Chinese remainder theorem looks like this:  
\begin{center}
1(mod 3)\\
1(mod 5) \\
4(mod 7) \\
\end{center}
The third system for the Chinese remainder theorem looks like this:  
\begin{center}
1(mod 3)\\
4(mod 5) \\
4(mod 7) \\
\end{center} 
Step 1: Find $M_{1}$, $M_{2}$ \& $M_{3}$, for each individual value of the modulo
\begin{center}
$M_{1}$ = 105 / 3 = 35 \\
$M_{2}$ = 105 / 5 = 21 \\
$M_{3}$ = 105 / 7 = 15
\end{center}
Step 2: Find the inverse of each of the M values above with there respective modulo and let them be dictated by $y_{1}$, $y_{2}$ , $y_{3}$  
\begin{center}
2 is the inverse of 35(mod 3) as 2*35 = 1(mod 3). Thus the value of $y_{1}$ is 2 \\
1 is the inverse of 21(mod 5) as 1*(21) = 1(mod 5). Thus the value of $y_{2}$ is 1 \\ 
1 is the inverse of 15(mod 7) as 1*(15) = 1 (mod 7). Thus the value of $y_{3}$ is 1
\end{center}
x = $a_{1}$$M_{1}$$y_{1}$ + $a_{2}$$M_{2}$$y_{2}$ + $a_{3}$$M_{3}$$y_{3}$ \\\\
For the first system solution the values of $a_{1}$ = 1, $a_{2}$ = 1 and $a_{3}$ = 3. Where x: \\ \\
= (1)(35)(2) + (1)(21)(1) + (3)(15)(1) \\ 
= 136 \\
136 $\equiv$ 31(mod 105) \\\\
For the second system solution the values of $a_{1}$ = 1, $a_{2}$ = 1 and $a_{3}$ = 4. Where x: \\ \\
= (1)(35)(2) + (1)(21)(1) + (4)(15)(1) \\ 
= 151 \\
151 $\equiv$ 46(mod 105) \\\\
For the third system solution the values of $a_{1}$ = 1, $a_{2}$ = 4 and $a_{3}$ = 4.Where x: \\ \\
= (1)(35)(2) + (4)(21)(1) + (4)(15)(1) \\ 
=  214\\\\
214 $\equiv$ 4(mod 105) \\\\
Therefore 3 of 8 solutions of the congruence of \textbf{ $x^2$ $\equiv$ 16(mod 105)} are \\ x = 31(mod 105), x = 46(mod 105), x = 4 (mod 105)

\newpage

\section{Question \#4. [10 Marks]}
Solve the congruence 2x = 7(mod 17) using the inverse of 2 modulo 17 \\\\
First, since the gcd (2,17) = 1, we know these numbers are relatively prime. \\
Next, we need to find the inverse of 2 modulo 17, this means 2*(some integer) = 1 (mod 17), and by inspection, we can determine that the integer must be 9, as 2(9) = 18 when divided by 17 gives remainder 1 thus 9 is the inverse of 2(mod 17) as 2*(9) = 1(mod 17)\\ \\ 
Next, we can multiply both sides of the equation by 9 to 2x = 7(mod 17) \\ 
\begin{center}
9*(2x) $\equiv$ 7*(9) (mod 17) \\
x $\equiv$ 63 (mod 17) \\ 
x = 12 \\ 
Therefore, all solutions of x are in the form 12 + 17n\\ where n is any real integer
\end{center}

\newpage


\section{Question \#5. [20 Marks]}

\newpage 
\section{Question \#6. [20 Marks]}

\newpage

\section{Question \#7. [30 Marks]}

\newpage

\section{Question \#8. [30 Marks]}

\newpage

\section{Question \#9. [30 Marks]}

\newpage

\section{Question \#10. [10 Marks]} 

\newpage

\section{Question \#11. [20 Marks]}

\newpage

\section{Questions \#12. [20 Marks]} 


\end{document}




















